\chapter{Summary}

The shift from enterprise integration to linked enterprise data
seems particularly suitable for the BBC given the legacy of
distinct data sources that has built up over time. It is hoped that
the search application can act as a strong example of the advantages
of publishing enterprise data as linked data and that integration
costs can significantly reduce as a result.

It was noted in section~\ref{sec:current-impl} that global search
engines such as Google create a pressure on websites to structure
their pages in certain ways and even to include semantic metadata. A
search application that makes this metadata the foundation of its
discovery and indexing processes (even if mixed with some direct integration
for data enrichment) might create additional pressures on BBC websites
to not only include more of their business data in public pages, but
also to link that data to related content in other sites.x

A search application that bases its indexing in linked data and
web standards should be more agile and this will open up opportunities
to innovate on features visible to the users more rapidly.

Future analysis is recommended to address auxiliary features such as
the autocomplete\cite{morville2010search} feature that appears when a user
is typing in
the search box. Much of this data is manually-maintained as is the case
with Editor's Choice. Future work should be able to look into a suitable
combination of automatic inference and manual curation to collate
possible autocomplete or autosuggest searches and links with the need to
maintain the whole list of potential completions by hand. This should follow
naturally from the work on Editor's Choice, but the autocomplete
improvements should
be designed after any lessons are learnt from the Editor's Choice development.

The search application should endeavour to publish its own information
as linked data as well, where data are created. Examples include publishing
curation of recommended links or perhaps the application could create
pages that link and aggregate content around a topic or concept. Such an
approach could have powerful implications for the search application being
at the centre of breaking down silos across the BBC and being a central
place for users to make horizontal journeys between related content.

Further studies, analysis and proposals after implementations of this proposal
are recommended to investigate the benefits of linking more data
between the content on different BBC sites.
