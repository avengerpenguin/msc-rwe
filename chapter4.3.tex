\section{Editor's Choice}

It is proposed there can be
a reduction in some of that manual curation through the use
of linked data and editorial staff can focus on curating
their own linked data. Editor's Choice could evolve into a more
generalised and automated subsystem. A metadata-driven curation system
could use common patterns in the data to promote orders of magnitude more
top links with a fraction of the current effort.

Thus the proposal is to maintain a linked data store in parallel
to the document-oriented search indexes, from which entities
and concepts themselves can be extracted and indexed to supplement
the search results centred around web pages.

The names or labels for these entities (as identified by
predicates such as \texttt{rdfs:label}) can be stored in
a search index distinct to the index of web pages and this
will enable the query time searching to perform a dual search
both for \emph{documents} that contain the given keywords,
but also \emph{things} that seem to match the same query.

This is superficially similar to the current behaviour in that
queries that match against concepts in a manually-curated
taxonomy will return so-called ``best links'' stored
against those concepts. These links then surface in the 
``Editor's Choice'' section of the search results page.

However, the accumulation of a single, large graph of
semantic metadata -- initially as a side effect of
indexing documents as their own distinct RDF graphs --
will allow this taxonomy to be grown over time without
manual intervention.

This does not preclude manual guidance and supervision of the
concepts that should surface in search results pages. In
the strictest model, the search system can simply suggest
entities that appear to be significant (e.g. if more
than a given number of pages and entities link to it) and an
editorial admin tool will allow approval of which items are
notable enough to match against queries. A more relaxed
model might see some entities with a high number of links
(for example, if a person or organisation rapidly becomes
famous in the news and is tagged in multiple news articles
as a result) being automatically included as potential
search results.

This follows the principles of linked enterprise data outlined in
section~\ref{linked-enterprise-data} in that staff within
an enterprise such as editorial staff in the BBC should be
creating data in a way that is automatically published. This
means that bespoke applications where the data are stored in
a database should be replaced over time with systems that
simply publish their curation as linked information, perhaps
directly into the search application's triple store.

Editorial staff might publish assertions such as
``person $X$ is notable for $Y$'' or
``programme $X$ is a big-name brand for the BBC'' and then
properties such as \texttt{foaf:homepage}\cite{brickley2012foaf} or
\texttt{po:microsite}\cite{raimond2009bbc} indicate the canonical
``home pages'' for those entities. The search application can
then index all entities that have such editorial interest marked
against them alongside their home pages and provide an
Editor's Choice feature that uses the combined strengths
of automated and manual curation.
