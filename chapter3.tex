\chapter{Analysis}

In this chapter, I will give a deeper analysis of the four
components of the search application identified in
chapter~\ref{intro} and start to highlight areas where linked data
approaches or semantic web technologies could have a r\^ole to play.

\section{Indexing Content}

\subsection{Overview}

The Search application needs to read in or receive metadata about
BBC content from a variety of sources and then needs to serve
a diverse set of use cases.\cite{fenning2014applicability} The
need to integrate against different data sources and funnel
that information into a common set of search indexes can
typically be seen as an \emph{enterprise integration}
problem.

Some of the data is fairly flat, textual content such as articles and some
information -- such as programmes -- is very structured (e.g. programmes,
series, episodes, times of broadcast, channel on which they are broadcast,
what times they are available to watch on iPlayer, what kinds of devices
are permitted by rights to watch them on iPlayer). There are challenges
in providing a consistent and meaningful user experience across such
a heterogeneous mix of content.

We can achieve some uniformity of the data by building a single, common
API through which all content is fed into the Search application. However,
some integration work needs to be
done by either the owners of the content to send content to the API
or the developers of the Search application, who can integrate
existing feeds via an adapter pattern.

Some pages on the BBC website no
longer have a dedicated team maintaining them and can sometimes only
be discovered via a web crawl as an external search engine would do. A lot
of metadata about content also lives in a diverse range of databases and
content management systems, which may or may not be easily retrievable
via the public web.

\subsection{Current Implementation}

\begin{dot2tex}[dot,options=-tmath]
digraph G {
  rankdir=LR

    Programmes -> Dynamite
    News -> CPS
    Sport -> CPS
    
    Dynamiate -> FileQueues
    CPS -> FileQueues

    FileQueues -> Transformation

    Transformation -> Multicast

    Multicast -> IndexMasters

    IndexMasters -> IndexSlaves


    }
\end{dot2tex}

\section{Search Results Page}

\section{Editor's Choice}

\section{Search Suggest}
