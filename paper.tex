\documentclass[10pt,a4paper]{report}

\usepackage{nag}

\usepackage{rotating}
\usepackage{graphicx}

\usepackage{mathtools}
\usepackage{natbib}
\usepackage{url}
\usepackage{verbatim}

\usepackage{tikz}
\usetikzlibrary{shapes,arrows}
\usepackage[autosize]{dot2texi}

\usepackage{xcolor}

 \usepackage{listings}

 \lstset{
         basicstyle=\footnotesize\ttfamily, % Standardschrift
         %numbers=left,               % Ort der Zeilennummern
         numberstyle=\tiny,          % Stil der Zeilennummern
         %stepnumber=2,               % Abstand zwischen den Zeilennummern
         numbersep=5pt,              % Abstand der Nummern zum Text
         tabsize=2,                  % Groesse von Tabs
         extendedchars=true,         %
         breaklines=true,            % Zeilen werden Umgebrochen
         keywordstyle=\color{red},
    		frame=b,         
 %        keywordstyle=[1]\textbf,    % Stil der Keywords
 %        keywordstyle=[2]\textbf,    %
 %        keywordstyle=[3]\textbf,    %
 %        keywordstyle=[4]\textbf,   \sqrt{\sqrt{}} %
         stringstyle=\color{blue}\ttfamily, % Farbe der String
         showspaces=false,           % Leerzeichen anzeigen ?
         showtabs=false,             % Tabs anzeigen ?
         xleftmargin=17pt,
         framexleftmargin=17pt,
         framexrightmargin=5pt,
         framexbottommargin=4pt,
         %backgroundcolor=\color{lightgray},
         showstringspaces=false      % Leerzeichen in Strings anzeigen ?        
 }
  \usepackage{caption}
\DeclareCaptionFont{white}{\color{white}}
\DeclareCaptionFormat{listing}{\colorbox[cmyk]{0.43, 0.35, 0.35,0.01}{\parbox{\textwidth}{\hspace{15pt}#1#2#3}}}
\captionsetup[lstlisting]{format=listing,labelfont=white,textfont=white, singlelinecheck=false, margin=0pt, font={bf,footnotesize}}


\colorlet{punct}{red!60!black}
\definecolor{background}{HTML}{EEEEEE}
\definecolor{delim}{RGB}{20,105,176}
\colorlet{numb}{magenta!60!black}

\lstdefinelanguage{json}{
    basicstyle=\normalfont\ttfamily\footnotesize,
    numbersep=8pt,
    showstringspaces=false,
    breaklines=true,
    literate=
     *{0}{{{\color{numb}0}}}{1}
      {1}{{{\color{numb}1}}}{1}
      {2}{{{\color{numb}2}}}{1}
      {3}{{{\color{numb}3}}}{1}
      {4}{{{\color{numb}4}}}{1}
      {5}{{{\color{numb}5}}}{1}
      {6}{{{\color{numb}6}}}{1}
      {7}{{{\color{numb}7}}}{1}
      {8}{{{\color{numb}8}}}{1}
      {9}{{{\color{numb}9}}}{1}
      {:}{{{\color{punct}{:}}}}{1}
      {,}{{{\color{punct}{,}}}}{1}
      {\{}{{{\color{delim}{\{}}}}{1}
      {\}}{{{\color{delim}{\}}}}}{1}
      {[}{{{\color{delim}{[}}}}{1}
      {]}{{{\color{delim}{]}}}}{1},
}

% Language Definitions for Turtle
\definecolor{olivegreen}{rgb}{0.2,0.8,0.5}
\definecolor{grey}{rgb}{0.5,0.5,0.5}
\lstdefinelanguage{ttl}{
    basicstyle=\normalfont\ttfamily\footnotesize,
    showstringspaces=false,
sensitive=true,
morecomment=[l][\color{grey}]{@},
morecomment=[l][\color{olivegreen}]{\#},
morestring=[b][\color{blue}]\",
}

\title{Proposal to improve BBC Search through the use of Linked Data}
\author{Ross Fenning}
\date{10 May 2014}

%Options: Sonny, Lenny, Glenn, Conny, Rejne, Bjarne, Bjornstrup
\usepackage[Bjarne]{fncychap}

\begin{document}

\maketitle

\tableofcontents

\chapter{Background}

This report proposes improvements to the BBC Search product making use
of \emph{Linked Data}\cite{bizer2009linked} and/or the Semantic
Web\cite{berners2001semantic}. I hope to be able to demonstrate five tangible
benefits this domain can bring to a search application, namely:

\begin{enumerate}
  \item Further decoupling of BBC applications, which aids scalability,
    maintainability and resilience in many software architectures.\cite{}
  \item Improvements to the search results page that help users determine
    if the results given are relevant via so-called
    \emph{Rich Snippets}\cite{goel2009introducing}
    that can automatically display some semantic information about a given
    result without the need for manual curation.
  \item User experience and design improvements to the search results
    where semantic properties about the results can drive visual decoration
    and calls to action, which can allow for a more flexible and
    adaptable design without the need to enumerate all ``kinds'' with
    different templates.
  \item Removal of the manual curation of the ``Editor's Choice'' results
    against a given query and replacing it with curated linked data that
    can provide recommended content against queries in a more general
    manner.
  \item Replacement of the manual curation of autocomplete suggestions
    in the \emph{Search Suggest} service -- that suggests full text queries
    as the user types -- with suggestions driven from linked data.
\end{enumerate}

In the rest of this chapter, I will give an overview of what role BBC
Search plays and how we can usefully view the system as four
distinct subsystems. In later chapters we can then look to improve these
four separate systems one at a time and then evaluate how these improvements
might complement each other.

\section{BBC Search}

BBC Search is an online product that forms part of the BBC website
to enable users to search for and navigate all areas of the site using
free text queries. This might aid visitors who are unable to find their desired
content via other global or site-specific navigation or those who are simply
looking for things that fall below the radar of the latest, greatest or
otherwise promoted content.

General-purpose search engines such as Google are also capable of directing
users to BBC pages via free text queries, but there is a benefit the BBC
can bring with its own search service that has greater knowledge of the
BBC's own content. The BBC can use internal knowledge to predict a user's
intent (e.g. if the query matches a programme title, then show links to watch
the latest episode on iPlayer) or to provide editorially-curated
links to significant pages (e.g. if a user searches for a term
related to a major, ongoing news theme, curated links can
link to an overview or FAQ for that ongoing event for those
looking for the background on it).

It is well established that a classical information retrieval system
can be expressed as two sets of computation: processing of information
into indexes (also known as \emph{indexing}) for quicker retrieval
and the retrieval of documents from those indexes based on a user's
query. These can be known as \emph{index time} or \emph{query time}
processing respectively.

The general strategy is to perform
any expensive transformation, expansion or ordering of data
at index time so that query time algorithms may be optimised to
run as quickly as possible. This is an approach that has only
increased in importance as information retrieval systems have moved
from standalone systems (e.g. a catalogue search situated within
a library used by a handful of users at once) to web-based
searches across a far larger corpus with tens or even hundreds of
users submitting queries in a single second.

In the remaining sections of this chapter, I outline the considerations
around these two sides of the search application in addition to the
two complementary systems known as \emph{Editor's Choice} and
\emph{Search Suggest}.

\section{Indexing Content}

The Search application needs to read in or receive metadata about
BBC content from a variety of sources and then needs to serve
a diverse set of use cases.\cite{fenning2014applicability} The
need to integrate against different data sources and funnel
that information into a common set of search indexes can
typically be seen as an \emph{enterprise integration}
problem. Architectural patterns have been developed, tested and
implemented in a large number of enterprises to synchronise,
transform and normalise data. Such patterns are notably
summarised by Hohpe and Woolf.\cite{hohpe2004enterprise}

Some of the data is fairly flat, textual content such as articles and some
information -- such as programmes -- is very structured (e.g. programmes,
series, episodes, times of broadcast, channel on which they are broadcast,
what times they are available to watch on iPlayer, what kinds of devices
are permitted by rights to watch them on iPlayer). There are challenges
in providing a consistent and meaningful user experience across such
a heterogeneous mix of content.

We can achieve some uniformity of the data by building a single, common
API through which all content is fed into the Search application. However,
some integration work needs to be
done by either the owners of the content to send content to the API
or the developers of the Search application, who can integrate
existing feeds via an adapter pattern.

Some pages on the BBC website no
longer have a dedicated team maintaining them and can sometimes only
be discovered via a web crawl as an external search engine would do. A lot
of metadata about content also lives in a diverse range of databases and
content management systems, which can be collectively described as part
of the so-called \emph{deep web}.\cite{}

In chapter~\ref{stateoftheart}, we will discuss how the deep web is one
problem the linked data domain seeks to solve.

\section{Retrieving Content}

Once we have constructed an index optimised for keyword-based retrieval,
we can begin to run queries against that index. A retrieval problem
can be summarised as trying to maximise two separate measures: \emph{precision}
and \emph{recall}, where:

\begin{displaymath}
  \textbf{precision} = \frac{
    |\text{relevant documents} \cap \text{retrieved documents}|
  }{
    |\text{retrieved documents}|
  }
\end{displaymath}

\begin{displaymath}
  \textbf{recall} = \frac{
    |{relevant documents} \cap \text{retrieved documents}|
  }{
    |\text{relevant documents}|
  }
\end{displaymath}

Namely, precision is the fraction of the retrieved documents that were
relevant to the user's query and recall is the fraction of documents out of
all possible, relevant documents that appear in the search results.

Note that a system could trivially maximise recall by always returning all
results, but such a system would perform poorly in terms of precision. A
measure known as \emph{F-measure} was derived by van Rijsbergen
\cite{rijsbergen1979information} which
balances precision recall based on preferring recall over precision $\beta$
times:

\begin{displaymath}
  F_\beta = (1 + \beta^2) \cdot \frac{\mathrm{precision} \cdot \mathrm{recall} }{ \beta^2 \cdot \mathrm{precision} + \mathrm{recall}}
\end{displaymath}

There are other measures such as average precision and average mean precision
that take into account ordering of results and the average over multiple
queries respectively. A successful algorithm or third party search engine
application can be evaluated quantitatively by these measures and more. A more
qualitative summary from the perspective of the end users could be:

\begin{quote}
  A successful search application will return only the content the user wants
  (precision) --
  as high up the results list as possible -- and as much of desired content
  as possible (recall).
\end{quote}

The set of all documents that could be retrieved by a query is known as the
\emph{corpus} and search systems will typically apply a \emph{score} against
each that gives some measure of confidence of the relevance of each document
given a particular query. It should be noted that scores are a relative
metric only and only have meaning within the context of a particular
retrieval algorithm and particular query.

\section{Editor's Choice}

The current BBC Search application returns an ordered list of results
for queries based on keyword matching within the titles, text or other
chosen metadata fields as appropriate. However, there are cases where
the algorithms employed may not return something noted as the official
or canonical match for a concept, person, programme, etc. (e.g. the
official microsite for a popular BBC programme or a profile page for
a notable person on BBC News).

Editorial staff thus have control over a curation system that manifests
as an \emph{Editor's Choice} feature that will ensure the top one to three
results for popular queries link to manually-chosen pages deemed as the most
relevant pages for those queries. For instance, a search for a city such
as \emph{Coventry} might naturally return recent news articles that mention the
city (since news results are biased towards more recent articles over older
ones), but the Editor's Choice links on top of the so-called \emph{organic
results} can persistently offer links for common use cases such as
the BBC Weather page for Coventry.

Each and every link promoted within this feature is hand-chosen by editorial
staff and much manual effort is needed to keep promoting new concepts
as they appear and to remove pages as they get superseded. It is hoped
in this proposal we can reduce some of that manual curation through the use
of linked data and demonstrate ways editorial staff can focus on curating
their own linked data such that Editor's Choice could evolve into a more
generalised and automated subsystem. A metadata-driven curation system
could use common patterns in the data to promote orders of magnitude more
top links with a fraction of the current effort.

\section{Search Suggest}

One challenge for the querying of a search application is to be tolerant
of the variable quality of user-entered queries. Many users of web search
engines may give ambiguously short queries, overly long queries with (where
it can be hard for a machine to pick out the important keywords amongst
less important words or stop words\cite{rajaraman2011data})
or simply use incorrect spelling.

This problem can be ameliorated with a plethora of algorithm
server-side solutions (e.g. spell checking), but a simple solution that
can be implemented on the user interface is an \emph{autocomplete}
or \emph{autosuggest} feature. The former prompts for potential
queries, titles, names, etc. that they may be typing so as to prevent
the need to type the query out in full. The latter feature can go
further to suggest onwards journeys that bypass the need for a
page of search results altogether.

The search box for BBC Search implements an autocomplete
feature -- known
as \emph{Search Suggest} -- whereby
the full names or titles of are suggested to users as they type the first
few characters of their queries. This helps guide users into free text
queries that are more likely to match terms in the indexes and reduces
mismatching due to misspellings or otherwise poorly-formed queries.

The list of suggestions that might surface in this system are
currently chosen and maintained by hand by editorial staff. If a new
concept breaks into the news (such as a notable person becoming notable
for the first time), then a person must enter that term into the system.
Similarly, if something should not be suggested to users any longer
(perhaps for legal reasons or simply because it is no longer a relevant
concept), then it must be amended or removed manually.

This manual maintenance of the suggestion list also leads to a
disconnect between searchable content and suggested queries that might
appear as users type. There is an onus on editorial staff to ensure
that content will be found when a suggested query is used. It might
serve the website users better to find mechanisms by which to generate
a useful set of suggestions from the corpus of content as it is
indexed. The heterogenous nature of BBC content turns this into
further integration work where different business rules are likely required
to generate suggestions from different types of content.

In chapter~\ref{stateoftheart} I will briefly outline some of the state of the
art in the Linked Data domain and start to highlight areas that show
the most promise for improving the four areas of the search application
described in this chapter.

\chapter{Linked Data}
\label{stateoftheart}

The concept of Linked Data can well be summarised as a representation of
data that follows four principles:\cite{berners2011linked}

\begin{enumerate}
  \item Use URIs to denote things.
  \item Prefer HTTP URIs (i.e. URLs) so that these things can be looked
    up (i.e. \emph{dereferenced}.
  \item When these URIs are dereferenced, provide useful information about
    those things using standards.
  \item Include links to other things (using their URIs) when such
    information is published on the web.
\end{enumerate}

It can be seen as a subset of Semantic Web techniques and it builds upon
the \emph{Resource Description Framework} (RDF)\cite{lassila1999resource},
which is a general-purpose method for describing and modelling information.

In this chapter, I will outline how web authors are encouraged to publish
linked data within their content and how that is already used to enhance
search results in global search engines. I will then introduce how linked
data may be used within an enterprise setting to reduce traditional
integration efforts and then touch on some other aspects of the linked
data domain that could be of interest to the search application.

\section{Publishing Linked Data within Web Pages}
\label{publishing-linked-data}

Whilst there exist multiple formats for publishing RDF purely for machines
(e.g. Turtle\cite{world2014rdf}, N-Triples, JSON-LD\cite{world2014json},
N3, RDF/XML) there are multiple ways to embed semantic or linked data within
an HTML web page:

\begin{itemize}
  \item RDFa\cite{adida2012rdfa} and RDFa Lite\cite{lite2004rdfa}.
  \item DC-HTML defines how to embed the Dublin Core vocabulary within HTML
    meta tags.
  \item Microdata is an HTML standard that allows annotating existing markup
    via additional attributes so that it serves as metadata in addition to
    its purpose as text for humans.
  \item Microformats are another mechanism that embeds semantic metadata
    within existing HTML tags and attributes.
\end{itemize}

Using one or more of these approaches allows a publisher to convey
machine-readable metadata in the same document that is served to humans.
This allows, for example, search engines to index more categorical and
structured information about a page when it is crawled in addition to the
normal processing on the text content. This was the foundation of the
Semantic Web\cite{berners2001semantic} before the term Linked Data
emerged and the aspiration was that widespread adoption would lead to
a comprehensive ``Web of Data'' to rival the rich, so-called
``Web of Documents'' that is the World Wide Web as we know it as a collection
of human-readable pages.

People such as Hepp have argued against adding markup such as RDFa around
the text visible to humans and instead adding hidden, additional markup to
contain semantic information.\cite{hepp2009rdf2rdfa} Hepp argues that
this allows the HTML tree structure to be decoupled from the data structure
of the vocabulary in use, although it does introduce redundancy if all
semantic information is duplicated in both human- and machine-readable forms.

Google have seemingly partially adopted an approach following this principle
-- at least for web authors publishing information about music
events.\cite{googlejsonld} Google are piloting embedding RDF metadata in
\emph{JSON-LD} format within HTML \texttt{<script>} tags to add semantic
information alongside textual data, but without annotating this metadata
around the textual content itself as we would with RDFa, for example.

All of the above approaches can be found ``in the wild'' in pages across
the Web with differing rates of adoption. General-purpose web search
engines can make use of this additional metadata to improve the quality
of the search results in a number of ways. In section~\ref{enhancing-results},
I discuss some of these potential improvements.

\section{Enhancing Search Engine Results}
\label{enhancing-results}

There are at least two key ways in which a search application might
improve the quality of results by making use of semantic metadata as
described in section~\ref{publishing-linked-data}.

One approach is to allow
user queries to be matched against metadata properties in addition
to the text, perhaps with higher precedence. This may lead to a search
engine that seems ``intelligent'' enough not just to retrieve articles
by keywords found therein, but also by keywords that match related
concepts and properties.

For example, a page about a film might be retrieved
by matching names of actors that appeared in the film, even if that actor's
name does not appear in any textual information -- e.g. the synopsis --
that appears on the page. If the author has published semantic metadata
within the the film's page to list all the actors and other contributors,
then our search application can match on this auxiliary information.

Whilst this technique could improve the relevance of the results for a
given query (and consequently improve the precision of the system for that
particular query), a second technique known as \emph{Rich Snippets} improves
results from a usability perspective by aiding users in assessing whether
a result is relevant or not.

Rich Snippets are a distinct improvement 
to the user experience and information architecture
of search applications. A search engine results page (sometimes known as a
\emph{SERP}) may well retrieve almost all relevant results for a query -- and
achieve nearly one hundred percent precision -- but this high performance
is lost on a human user if the results are presented in a form where it is
hard to determine the respective relevant of each result.

Evaluation of the effectiveness of search results enhanced with rich snippets
via both explicit and implicit user feedback had been carried out by
Haas et al.\cite{haas2011enhanced} at Microsoft and Yahoo! research. The
results showed a promising preference from 84\% of users for a result
with enhanced information with the most common downside being that the
snippet may sometimes present incorrect information (e.g. showing the wrong
price for a product is less preferable than simply not attempting to present
the price at all). Implicit feedback via click-through rates (CTR) --
a common metric to indicate that users are happy enough with the information
given to click on the link given -- suggested users were more likely to
follow enhanced results.

A higher CTR for semantically-enhanced search results creates some pressure
and incentive on web publishers to include semantic metadata in their web
content if they aspire to attacting visitors via global search engines such
as Google, Bing and Yahoo!.

\section{Linked Enterprise Data}
\label{linked-enterprise-data}

\emph{Enterprise integration} is an important technical field within the
discipline of Enterprise Architecture that focuses on interconnecting
systems and allowing data to flow between them. This can be seen as
the practice of breaking down information silos that naturally emerge
in enterprise organisations.\cite{allemang2010semantic}

Architectural patterns have been developed, tested and
implemented in a large number of enterprises to synchronise,
transform and normalise data. Such patterns are notably
summarised by Hohpe and Woolf.\cite{hohpe2004enterprise}

There are difficulties with some of these approaches noted by
Allemang\cite{allemang2010semantic}. Data warehousing,
integration projects and metadata repositories can all
be time-consuming to implement, require repeated updating when
use cases change and at the extreme can be seen as solving the
problem of too many silos by adding another silo.

In recent years, a growing trend has emerged to use
Linked Data in the enterprise to remove the need
for repeated integration projects or data warehousing.
indexes over bespoke integration efforts. This follows from
a major motivation of Linked Data being that it aids in breaking
down information silos.\cite{bizer2009linked} In the
\emph{Linked Enterprise Data} approach, all information
production remains \emph{distributed} but is
\emph{connectable}.\cite{allemang2010semantic}

An organisation that seeks to ensure that all information is
published as Linked Data can learn from examples such as Wikipedia
-- or indeed the web in general -- where individual experts are
encouraged to share knowledge rather than keep it in silos. This
distributed approach recognises that diversity and flux are the
steady state of an agile enterprise. Two sources that happen
to use the same vocabulary are instantly interoperable. Two
sources that use a different vocabulary could well be integrated
simply by creating appropriate mappings between the vocabularies.
It should not be necessary to re-engineer a whole dataset nor
create entirely new systems just for integration.

There are still costs however in additional services, infrastructure
and training in exposing data as linked data throughout the
enterprise.\cite{hyland2010preparing} A practical approach
has been suggested by Davis\cite{davis2011achieving} where enterprises
can expose data via known formats such as Atom\cite{nottingham2005atom}
and RESTful web services. Davis argues that the \texttt{atom:link}
element offers some of the functionality of an RDF triple where
an entity can be linked to other entities via the \texttt{href}
attribute and the nature of that relationship is described by
the \texttt{rel} attribute.

The use of RESTful services and XML formats are very common in
the enterprise -- noting however that many do not follow all tenets
of the REST architectural style. This suggests that businesses
could adopt the pattern suggested by Davis to expose their
data across the organisation using technologies and techniques
with which they are likely familiar. This eschews the full power
of a stack based RDF, OWL and SPARQL, but might be a better
opportunity for businesses that otherwise would not take on a
Linked Data approach at all.  Many proponents of Linked Data would
argue that the important aspect is that the data are exposed and
linked, however that is achieved.

\begin{comment}
If we view the indexing processes
of a search application as a variant of data warehousing,
we can start to argue for enterprise-wide publishing of
linked data as a more cost effective means of building search

An important consideration is that BBC Search is largely directing
users to public pages on the BBC website, a large number of which have
already published some of their metadata as linked data. Notably, BBC News,
BBC Sport and BBC Food all use some combination
of microformats, microdata, RDFa, HTML meta tags within their HTML content
pages. Some of this was to improve search results in external search engines
such as Google and some of it ties into Facebook's Open Graph Protocol, which
enables and enhances social media sharing of the pages.

These efforts are not as structured and rich as the more notable uses
of semantic web technologies on the programmes microsites, Nature and BBC
Music \cite{raimond2010use}, but indicate that some level of metadata
can be retrieved
as linked data. In cases where the metadata are rich enough, it might be
possible to avoid direct integration with databases, hidden back end
web services and other structured ``feeds'' typically set up for enterprise
integration.
\end{comment}


\chapter{Analysis}

In this chapter, I will 

\section{}

\chapter{Proposal}

In chapter~\ref{analysis}, the current state of BBC Search was outlined
along with the maturity of linked data and semantic web technologies
across BBC online content. In this chapter, I will propose tangible,
potential improvements to the search application that arise from
the use of linked data.

\section{Linked Data Indexing}

The concept of \emph{Linked Enterprise Data} was introduced and described
in section~\ref{linked-enterprise-data}. In this section, it is proposed
to endeavour to make use of this approach where possible in the
ingest and indexing processes of the BBC Search application.

\begin{figure}
  \begin{center}
\begin{dot2tex}[dot,scale=0.55]
digraph ingest {
  rankdir=TB;

  Programmes [shape=rect,label=iPlayer]
  News [shape=rect]
  Newsround [shape=rect]
  Newsbeat [shape=rect]
  Sport [shape=rect]
  MarketData [shape=rect]
  WorldService [shape=rect]
  BBCFood [shape=rect]
  ProgrammesWebsites [shape=rect,label=Programmes]
  LegacyWebsites [shape=rect,label="Legacy Sites"]

  SearchIndexes [shape=circle]

  Programmes -> URLNotification [label="URLs"]

  News -> CPS
  Newsround -> CPS
  Newsbeat -> CPS
  Sport -> CPS  

  CPS -> URLNotification [label="URLs"]

  URLFetch -> "www.bbc.co.uk" [label="HTTP\ GET"]
  "www.bbc.co.uk" -> URLFetch [label="RDFa/Microdata"]

  trans3 [label="Transformation"]
  WorldService -> trans3
  MarketData -> trans3
  trans3 -> IngestAPI [label="Atom"]
  
  ProgrammesWebsites -> Crawler
  BBCFood -> Crawler
  LegacyWebsites -> Crawler

  Crawler -> URLNotification [label="URLs"]

  URLNotification -> URLFetch [label="RDF"]

  URLFetch -> Enrichment [label="RDF"]
  URLFetch -> Inference [label="RDF"]

  Enrichment -> Inference [label="RDF"]

  Inference -> IngestAPI [label="RDF"]

  IngestAPI -> SearchIndexes

}
\end{dot2tex}
  \end{center}
  \caption{Abstract model of proposed indexing process}
  \label{led-indexing-chain}
\end{figure}

An abstract overview of the flow of content is given in figure~\ref{led-indexing-chain}.
As with the current system depicted in figure~\ref{indexing-chain},
there remains an ``Ingest'' API to which any system may
send content to be indexed if it has already been transformed to
use the semantics required by the search application. This
ensures that integration work done by other teams continues to work
and retains the use case where content producers wish to have full
control of what metadata is sent to the search indexed -- as long as
they are willing to do the integration work.

Bespoke integration in this way can still suffer from maintainability
problems that occur when systems are too tightly coupled. Due to this,
it is architecturally important that the Ingest API continues to follow
the REST architectural style, making appropriate use of Hypermedia to
evolve the API's semantics over time whilst not breaking existing clients.

Where this proposal differs from the existing approach is in the case
of more ``light touch'' integration for teams who are less able or willing
to do bespoke transformation work and increasing coupling to the search
APIs. The goal of this part of the proposal is to reduce the integration
and consequent maintenance work for the search development team.

For these cases, the indexing chain starts from the \emph{URLNotification}
stage to which messages can be sent in
\texttt{text/uri-list}\cite{amundsen2011hypermedia} format. It is expected
that there is some mechanism by which this service can be notified with URLs
of every piece of content that has been created or updated on the BBC website.
This might take the form of the search team's own crawler that is seeking
out pages that have not been found yet or it could be a message-driven
notification process. It is hoped that whilst it might be demanding to
put the onus on a team
to provide transformed content and to integrate against the search system's
APIs, it is a far lighter piece of work to send simple notifications containing
a URL.

The URLNotification service can generate a minimal RDF graph and the
rest of the indexing chain can be seen as a process that incrementally
augments this graph until it is suitable for indexing and has sufficient
information to display search results. It is proposed that these minimal
graphs will take the form of the following example (in JSON-LD serialisation):

\begin{centering}
\begin{lstlisting}[language=json]
[
  {
    "@id": "http://www.bbc.co.uk/news/technology-27346567",
    "http://purl.org/dc/terms/identifier": [
      {
        "@value": "http://www.bbc.co.uk/news/technology-27346567"
      }
    ]
  }
]
\end{lstlisting}
\end{centering}

This single triple, whilst arguably tautological, can serve as a stub
for the series of augmentations further downstream. The next stage,
the \emph{URLFetch} stage, is
the attempt to add semantic metadata found by dereferencing the URL
given. This step should retrieve metadata similar to the graphs depicted
in chapter~\ref{analysis} such as figure~\ref{keepod}. If the identifier
in the stub RDF graph sent to this stage is a URI, but not a URL, then
dereferencing is not possible, but the \emph{URLFetch} stage could
passively send the graph onwards without augmenting to it.

In the URLFetch stage, it might be necessary to apply some web scraping/crawling
techniques to extract the body of text from articles where appropriate.
It was identified in the case of figure~\ref{keepod} that BBC News
was not semantically highlighting the actual text content of the articles, but
a search application is certainly interested in this information.

The URLFetch service can then send the augmented graph onwards to
the \emph{Enrichment} service. The purpose of this stage is to apply
any number of business rules to add further metadata to the graph
that has not been obtainable via the public website or further upstream. It is
in this service that the search team is doing more classical
\emph{enterprise integration} work, which is unavoidable, but confined
to this one service (or collection of microservices).

In the Enrichment service, we might, for example, contact backend
service applications for futher data about the content we are trying to ingest.
This might mean fetching information about a programme from services
behind BBC iPlayer or contacting content management systems for
non-published metadata about articles, videos, etc.

Depending on the nature of the data source, this might allow us to start
making \emph{closed world} assumptions about the properties of content
items. In section~\ref{programmes}, it was noted that the \emph{open world}
assumption of linked data would make it difficult to assert negative
things about the content (e.g. this programme is not available to
watch on iPlayer). If we know that our RDF graph has been enriched
with a totality of information \emph{from source}, then it might be
possible to start making some closed world assumptions with certain predicates.

Following the same pattern as the URLFetch stage, it is valid behaviour
for the Enrichment service to assess that it is unable to provide
any useful, additional metadata and to pass the RDF graph further
downstream untouched. This allows an iterative approach where the first
version of the service can only enrich from one data source and
further data sources are added over time. Content items that cannot be
enriched at the present time can simply pass through unchanged.

The next stage is to run the RDF graph through an \emph{Inference} process.
This is where custom rules are invoked to handle the different
vocabularies that might be in use. For example, the rule shown below
(in N3 format) would allow us to infer that we can
use \texttt{schema:headline} anywhere we have used \texttt{rnews:headline}
or \texttt{og:title}.

\begin{centering}
\begin{lstlisting}[language=ttl]
@prefix schema: <http://schema.org/> .
@prefix rnews: <http://iptc.org/std/rNews/2011-10-07#> .
@prefix og: <http://ogp.me/ns#> .

{ ?x rnews:headline ?y } => { ?x schema:headline ?y } .
{ ?x og:title       ?y } => { ?x schema:headline ?y } .
\end{lstlisting}
\end{centering}

This is useful if the decision is made to use the Schema.org vocabulary as
the standard within the search indexes and application, but upstream
content from BBC News is using \texttt{rnews:headline} instead. It is this
mixed used of vocabularies that was a major driver in historic
enterprise integration efforts. In bespoke integration applications, explicit
transformation would be used to translated  ``news item'' to a
``search item'' including the rule that explicitly extracted the
\texttt{rnews:headline} property and set it on the \texttt{schema:headline}
property on the new ``search item'' being created.

In the proposed approach, the RDF model provides a generic framework where
transformation does not need to occur at any syntactic level and we
can use resuable inference rules to make more blanket statements such as
``Where the title has been set for Facebook's Open Graph, use the same
thing for the title on search results''. This rule can then apply to \emph{all}
content that uses that property without regard to what ``kind'' of content item
it is.

This may sound too open to problems where properties are miused in
unexpected ways and that inferring information in key vocabularies from
ones used by other systems might allow for ``dirty data'' to appear. Such
difficulties with the variable quality of linked data on the wider semantic
web were certainly encountered by Hogan et al.\cite{hogan2011searching} when
building the SWSE: the Semantic Web Search Engine. It is argued however,
that these risks are substantially lower for linked enterprise data as
the domain is far more controlled than the public web. It is also an important
feature of the web (both the document web and the semantic web) that linking and
decentralisation bring overall maintainability and interoperability benefits
that outweigh minor integrity concerns that organisations tradtionally attempt
to prevent with strict, controlled vocabularies, governance and bespoke
software integration.

This proposal argues it is more beneficial to take a more agile approach
of validating some minimum requirements of the information before it
enters the search indexes themselves. Any content that is not thus
being accepted can be flagged so that the rules can be amended. For example,
it might be decided that the \texttt{rnews:headline} property is only
correctly used by BBC News and other content producers use it incorrectly. In
this case, we can update our rules as shown below:

\begin{centering}
\begin{lstlisting}[language=ttl]
@prefix schema: <http://schema.org/> .
@prefix rnews: <http://iptc.org/std/rNews/2011-10-07#> .
@prefix og: <http://ogp.me/ns#> .

{ ?x rnews:headline ?y
. ?x og:site_name "BBC News" } => { ?x schema:headline ?y } .
{ ?x og:title             ?y } => { ?x schema:headline ?y } .
\end{lstlisting}
\end{centering}

Here we add an additional constraint
that we only infer the \texttt{schema:headline} property that we might
ultimately use on the search results page from the \texttt{rnews:headline} if
the content has a value for \texttt{og:site\_name} that indicates it came
from BBC News. This can then be iterated on for future rules that define
different behaviour based on different aspects of the content. These inference
rules can be as simple or as complex as needed.

It is even possible to have
completely different sets of rules invoked based on the types of the content
being indexed. Such an approach should be taken with care so as to prevent
being overly brittle and coupled to the exact structure of the content. More
generic rules allow us to add information based on abstract features of the
metadata that might be present across many ``types'' of content.


\chapter{Implementation Plan}

\chapter{Summary}

The shift from enterprise integration to linked enterprise data
seems particularly suitable for the BBC given the legacy of
distinct data sources that has built up over time. It is hoped that
the search application can act as a strong example of the advantages
of publishing enterprise data as linked data and that integration
costs can significantly reduce as a result.

It was noted in section~\ref{sec:current-impl} that global search
engines such as Google create a pressure on websites to structure
their pages in certain ways and even to include semantic metadata. A
search application that makes this metadata the foundation of its
discovery and indexing processes (even if mixed with some direct integration
for data enrichment) might create additional pressures on BBC websites
to not only include more of their business data in public pages, but
also to link that data to related content in other sites.x

A search application that bases its indexing in linked data and
web standards should be more agile and this will open up opportunities
to innovate on features visible to the users more rapidly.

Future analysis is recommended to address auxiliary features such as
the autocomplete\cite{morville2010search} feature that appears when a user
is typing in
the search box. Much of this data is manually-maintained as is the case
with Editor's Choice. Future work should be able to look into a suitable
combination of automatic inference and manual curation to collate
possible autocomplete or autosuggest searches and links with the need to
maintain the whole list of potential completions by hand. This should follow
naturally from the work on Editor's Choice, but the autocomplete
improvements should
be designed after any lessons are learnt from the Editor's Choice development.

The search application should endeavour to publish its own information
as linked data as well, where data are created. Examples include publishing
curation of recommended links or perhaps the application could create
pages that link and aggregate content around a topic or concept. Such an
approach could have powerful implications for the search application being
at the centre of breaking down silos across the BBC and being a central
place for users to make horizontal journeys between related content.

Further studies, analysis and proposals after implementations of this proposal
are recommended to investigate the benefits of linking more data
between the content on different BBC sites.


\bibliographystyle{plain}
\bibliography{bibtex}

\end{document}
