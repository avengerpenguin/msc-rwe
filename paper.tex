\documentclass[10pt,a4paper]{report}

\title{Proposal to improve BBC Search through the use of Linked Data}
\author{Ross Fenning}

\begin{document}

\maketitle

\chapter{Background}

This report proposes improvements to the BBC Search product making use
of \emph{Linked Data} \cite{} and/or the Semantic Web \cite{}. I hope
to be able to demonstrate five tangible benefits this domain can
bring to a search application, namely:

\begin{enumerate}
  \item Further decoupling of BBC applications, which aids scalability,
    maintainability and resilience in many software architectures. \cite{}
  \item Improvements to the search results page that help users determine
    if the results given are relevant via so-called \emph{Rich Snippets}
    that can automatically display some semantic information about a given
    result without the need for manual curation.
  \item User experience and design improvements to the search results
    where semantic properties about the results can drive visual sugar
    and calls to action, which can allow for a more flexible and
    adaptable design without the need to enumerate all ``kinds'' with
    different templates.
  \item Removal of the manual curation of the ``Editor's Choice'' results
    against a given query and replacing it with curated linked data that
    can provide recommended content against queries in a more general
    manner.
  \item Replacement of the manual curation of autocomplete suggestions
    in the \emph{Search Suggest} service -- that suggests full text queries
    as the user types -- with suggestions driven from linked data.
\end{enumerate}

In the rest of this chapter, I will give an overview of what role BBC
Search plays and how we can usefully view the system as a whole as four
distinct subsystems. In later chapters we can then look to improve these
four separate systems one at a time and then evaluate how these improvements
might complement each other.

\section{BBC Search}

BBC Search is an online product that forms part of the BBC website
to enable users to search for and navigate all areas of the site using
free-text queries. This might aid visitors who are unable to find their desired
content via other global or site-specific navigation or those who are simply
looking for things that fall below the radar of the latest, greatest or
otherwise promoted content.

General-purpose search engines such as Google are also capable of directing
users to BBC pages via free text queries, but there is a benefit the BBC
can bring with its own search service that has greater knowlege of the
BBC's own content. The BBC can use internal knowledge to predict a user's
intent (e.g. if the query matches a programme title, then show links to watch
the latest episode on iPlayer) or to provide editorially-curated
links to significant pages (e.g. if a user searches for a term
related to a major, ongoing news theme, curated links can
link to an overview or FAQ for that ongoing event for those
looking for the background on it).

It is well established that a classical information retrieval system
can be expressed as two sets of computation: processing of information
into indexes (also known as \emph{indexing}) for quicker retrieval
and the retrieval of documents from those indexes based on a user's
query. These can be known as \emph{index time} or \emph{query time}
processing respectively.

The general strategy is to perform
any expensive transformation, expansion or ordering of the indexed
at index time so that query time algorithms may be optimised to
run as quickly as possible. This is an approach that has only
increased in importance as information retrieval systems have moved
from standalone systems (e.g. a catalogue search situated within
a library used by a handful of users at once) to web-based
searches across a far larger corpus with tens or even hundreds of
users submitting queries in a single second.

In the remaining sections of this chapter, I outline the considerations
around these two sides of the search application in addition to the
two complementary systems known as \emph{Editor's Choice} and
\emph{Search Suggest}.

\section{Indexing Content}

The Search application needs to read in or receive metadata about
BBC content from a variety of sources and then needs to serve
a diverse set of use cases. \cite{fenning2014applicability} The
need to integrate against different data sources and funnel
that information into a common set of search indexes can
typically be seen as an \emph{enterprise integration}
problem. Architectural patterns have been developed, tested and
implemented in a large number of enterprises to synchronise,
transform and normalise data. Such patterns are notably
summarised by Hohpe and Woolf\cite{hohpe2004enterprise}.

Some of the data is fairly flat, textual content such as articles and some
information -- such as programmes -- is very structured (e.g. programmes,
series, episodes, times of broadcast, channel on which they are broadcast,
what times they are available to watch on iPlayer, what kinds of devices
are permitted by rights to watch them on iPlayer). This has challenges
for providing a consistent and meaningful user experience across such
a heterogenous mix of content.

% Do we need this?
Concepts central to the information retrieval domain such as creating
tokenisation, stemming, lemmatisation and relevance
algorithms can normally be delgated to
third party proprietary or free software. It is still useful however
for developers of full search application to be aware of these concepts even
if their implementation is abstracted away. Tokenisation, stemming and
lemmatisation are index time concerns, the full definitions of which
are out of scope for this report. In short, we want a search application
to understand individual words within an article and to understand that, for
example, ``mice'' and ``mouse'' are approximately the same word as well
as verb forms like ``swim'', ``swam'', ``swum'' and ``swimming'' all being
equivalent in the context of keyword retrieval.

\section{Retrieving Content}


\section{Editor's Choice}

\section{Search Suggest}

\chapter{State of the Art in Linked Data}

\section{Linked Enterprise Data}

In recent years, a growing trend has emerged to use
\emph{Linked Data} in the enterprise to remove the need
for repeated integration projects or data warehousing.
\cite{allemang2010semantic} If we view the indexing processes
of a search application as a variant of data warehousing,
we can start to argue for enterprise-wide publishing of
linked data as a more cost effective means of building search
indexes over bespoke integration efforts.

An important consideration is that BBC Search is largely directing
users to public pages on the BBC website, a large number of which have
already published some of their metadata as linked data. Notably, BBC News,
BBC Sport and BBC Food all use some combination
of microformats, microdata, RDFa, HTML meta tags within their HTML content
pages. Some of this was to improve search results in external search engines
such as Google and some of it ties into Facebook's Open Graph Protocol, which
enables and enhances social media sharing of the pages.

These efforts are not as structured and rich as the more notable uses
of semantic web technologies on the programmes microsites, Nature and BBC
Music \cite{}, but indicate that some level of metadata can be retrieved
as linked data. In cases where the metadata are rich enough, it might be
possible to avoid direct integration with databases, hidden back end
web services and other structured ``feeds'' typically set up for enterprise
integration.

\chapter{Analysis}

\chapter{Proposal}

\chapter{Implementation Plan}

\chapter{Summary}

\bibliographystyle{cell}
\bibliography{bibtex}

\end{document}
