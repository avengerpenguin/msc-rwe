\section{Features}

Whilst most of the proposals in chapter~\ref{ch:proposal} were
technical in nature, most Agile practices encourage phrasing
improvements in terms of changes a user or stakeholder would see
and attach some measure of business value to them. Thus the proposed
improvements around indexing linked data, semantically-driven rich
snippets and semi-automated editorial curation are expressed as
the follow tangible improvments:

\begin{itemize}

\item \textbf{A user can always find the microsite for any programme.}
  This removes the redundancy
  of crawling programmes microsites and indexing iPlayer as separate categories.

\item \textbf{A user has the choice to click through to iPlayer instead of a programme microsite if that programme has episodes available to watch.}
  This turns programmes results into richer
  snippets where a link to iPlayer becomes optional decoration based on whether that information
  is available. If search does not display the link, but it is actually available to watch, the
  programme's microsite will still link to iPlayer, simply adding an extra level of indirection. This
  is arguably preferable to omitting the programme from an ``iPlayer'' category because BBC Search
  (incorrectly) thinks it is not available to watch on demand.

\item \textbf{Search results for programmes should display when that programme is next broadcast.}
  This supports the use case where a user is looking for a ``quick answer'' as to when
  their chosen programme is next broadcast. User testing should contribute to ascertaining how valuable
  this use case is.

\item \textbf{A programme's microsite should always be the top result.}
  In a purely-organic search, it
  is possible that news articles, etc. might contain the same keywords as a programme (e.g. some programm
  titles are incredibly generic, such as \emph{Today} on BBC Radio 4) and those results might conceivably
  be scored more highly. The Editor's Choice feature ensures the programme's microsite is always at
  the top, but with this feature we can migrate the manual curation of these programmes links to
  the proposed, automated system and reduce the need to manually manage this subset of curated links.

\item \textbf{The search results for CBBC and Knowledge and Learning should be improved.}
  These are two BBC sites that are actively maintained, but content is currently obtained by
  the search application via a web crawler. The crawler can be retained for now, but it should
  be sending the content via the new RDF-based indexing chain laid out in section~\ref{sec:ld-indexing}.
  This will give the opportunity to extract some semantic data (CBBC at least uses Facebook
  Open Graph metadata) which is an improvement over using things like the \texttt{<title>} tag. The
  display of results from these two areas should improve slightly and there may be other sites
  that can receive these improvements as well.

\item \textbf{The search results page should be fully responsive.} The current search results page
  is implemented separately for desktop and mobile. A fully-responsive page would modernise
  the front end, provide opportunities to start incorporating further features around rich snippets
  and give an aesthetic veneer to the more data-driven improvements that are to come.

\end{itemize}

It should be noted that these are \emph{proposed} features. It is fully expected that
most will require specific analysis, prototyping or user validation. The presence of a given
feature on the road map indicates that it will be proven quickly (and fail quickly if it is going to fail)
as early on in its implementation as possible. The understanding is we are willing
to throw away ideas that do not ultimately work.
