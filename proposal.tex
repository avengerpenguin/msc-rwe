\documentclass[10pt,a4paper]{article}

%\usepackage{fullpage}

\title{Proposal to explore the feasibility of improving BBC Search through the use of Linked Data}
\author{Ross Fenning}

\begin{document}

\maketitle
\thispagestyle{empty}

\section{Introduction}

BBC Search is a online product that forms part of the BBC's online presence so
that users can search for and navigate all areas of the BBC website using
free text queries. This might aid visitors who are unable to find their desired
content via other global or site-specific navigation or those who are simply
looking for things that fall below the radar of the latest, greatest or
otherwise promoted content.

General-purpose search engines such as Google are also capable of directing
users to BBC pages via free text queries, but there is a benefit the BBC
can bring with its own search service that has greater knowlege of the
BBC's own content. The BBC can use internal knowledge to predict a user's
intent (if the query matches a programme title, then show links to watch
the latest episode on iPlayer) or to provide editorially-curated
links to significant pages (e.g. if a user searches for a term
related to a major, ongoing news theme, curated links can
link to an overview or FAQ for that ongoing event for those
looking for the background on it).

\section{The Problem}

The BBC Search service relies on an underlying search engine to support
full-text querying, which provides pages of results in a manner not
too dissimilar to Google. Internal BBC knowledge allows the results
to be broken up into logical ``zones'', but those full-text results
(sometimes referred to as the \emph{Organic Results} are matched
just using traditional Information Retrieval ranking
algorithms and linguistic techniques.

The search pages are therefore augmented with features such as
``Editor's Choice'' where text matching is done against a separate
system to provide manually-curated ``best links'' for a particular
search terms. This provides a useful bridge to recommended pages based
on precise concepts the user might be seeking and gives the system
the appearance of understanding that the user was seeking those
particular concepts rather than just pages that happen to contain
the same text.

For example, a search for ``manchester'' gives
us an Editor's Choice link for the index page for Manchester local
news within BBC News. This is a top link above any matches for pages
that happen to mention Manchester.

This is a useful feature which could be compared to Google's
`Knowledge Graph'', but falls short due to the need for all links
to be curated by hand against a manually-constructed taxonomy of terms.

The Editor's Choice feature is also limited only to matching a
concept by title to a recommended page link. There is no semantic
representation of the given concept that might allow richer linking
to multiple pages. For example, a search for an actor's name might
provide a recommended profile page within an Editor's Choice system,
but a deeper understanding of the person they are searching for might
allow automatic aggregation of recent news mentioning that actor and
all BBC programmes past and future in which they appear. Recommended
and curated links could still form part of these links, but there
is much that could be done with automatically linking entities and pages.

\section{Proposal}

The proposal is to conduct an analysis and feasibility study of how
the Search product could benefit from \emph{Linked Data} and how
it has been applied in other parts of the BBC and perhaps globally.

The analysis will look not only at what linked and semantic data is already
present in different pieces of content across the BBC that Search is
indexing, but also might provide proposals for how greater adoption
of Linked Data concepts might enable Search to provide better discovery
of that content.

The provision of better results might be achieved through use of additional
metadata linked to content items or the search results page might
at least be able to decorate relevant -- as determined by the
full-text search engine -- results with richer onward links, visual iconography
or calls to action.

The study may be able to provide automated replacements for
manually-maintained systems such as ``Editor's Choice'' or manual
tagging of content. Known synonyms for concepts that have to be fed into
the Search application from Editorial teams could be sourced automatically.
Where manual curation and review of content must take place, automated
tools might be able to assist and suggest information, which would
speed up such manual workflows.

\end{document}
