\section{Strategy and Technology}

As depicted in figure~\ref{led-indexing-chain}, the architectural
approach is based around discrete, decoupled, RESTful services
forming a conceptual pipeline that takes notification of content
through a process of augmentation, validation and indexing. Recognising
and exploiting Conway's Law\cite{conway1968committees}, a good strategy
would be to focus subteams on each major area of the application
(so that the team organisation fits the architecture rather than the
architecture emerging from the team structure).

Features will cut vertically down the entire technology stack, but
the individual engineering tasks to implement them can be split
across subteams focused on each of:

\begin{itemize}
  \item \textbf{Discovery.} This subteam will focus on the URLNotification
    and URLFetch services as well as any integration work upstream to
    generate notifications where such mechanisms do not already exist for certain
    types of content. Typical considerations will be around RSS/Atom feeds that
    might notify of new content, message-based notifications from other
    systems and maintaining a crawler that finds the rest.
  \item \textbf{Enrichment and Inference.} Here the development team will focus
    on integration with BBC or external data sources to enrich information
    coming from the discovery stages. The developers will also need
    to gain sufficient domain knowledge of each content area to make sound
    inferences from the enriched metadata that then generate sensible
    values for fields required by the search indexes.
  \item \textbf{Ingest.} The Ingest API and the search indexes themselves
    will be maintained by a subteam responsible for accepting, validating,
    archiving, indexing and re-indexing content. The Ingest API will have
    not just the enrichment stages as a client, but also any BBC team that
    wishes to push its data directly to the search system already transformed.
    Received content will need to be archived in a ``raw'' form so that
    indexes can be reorganised and rebuilt if their structure need to change. It
    is also possible that the archived content might be re-pushed via the
    enrichment stages to take advantage of new business rules added therein by
    the previous subteam. There is also much work just around designing the
    structure of indexes, configuring linguistic matching and otherwise
    optimising retrieval that it might prove necessary to split out a subteam
    dedicated to those tasks.
  \item \textbf{Results Pages.} This is the subteam dedicated to the front
    end web pages. This is where user experience, design and other front
    end concerns are a primary focus. It is expected that this subteam
    will make requirements for the ingest subteam.
\end{itemize}

The boundaries between subteams do not need to be strict and the team
itself can still behave as a single team sharing ideas, development practices
and tooling. The intention here is to have one or two developers that naturally
gravitate to being responsible for their nominated area so as to encourage
each area to behave in a decoupled way that is built to tolerate failure of
other parts of the system.

The loose coupling afforded by linked data and REST should allow these areas
to evolve independently without changes causing breakages. For example, the
discovery component should be able to generate more linked data through finding
new ways of extracting metadata from content without the enrichment stage
suffering. This in contrast to more brittle RPC-over-HTTP architectures where
an upstream client sending a new XML or JSON format would cause a service to
start rejecting the requests unless the service was updated to understand these
new requests.

Introducing some devops\cite{huttermann2012devops}
responsibilities to the individual subteams and
making use of cloud technology could even permit each subteam to
develop their respective component(s) in different programming languages or
with different frameworks since the components only share information
via HTTP. These are decisions to be made by the developers as they come
to evaluate the best tools for their respective tasks.
