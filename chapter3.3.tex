\section{Editor's Choice}

Three different categories of search activities were proposed
as part of a model developed by
Gary Marchionini\cite{marchionini2006exploratory}:
\emph{Lookup}, \emph{Learn} and \emph{Investigate}. The
latter two can be grouped under the banner of \emph{exploratory search}
and the diverse spread of organic results on the search results pages
can aid these activities. It is to serve the \emph{Lookup} activity
that the so-called organic results are header with up to three
``Editor's Choice'' links that -- based on the query given -- may
well lead directly to a specific item or location the user was seeking.

Two such distinct modules on the page allow both lookup- and exploration-
style interactions to occur on the same results page. It could be said
that the Google Knowledge Graph\cite{singhal2012introducing} takes
this concept further to provide summaries of entities and links to
specific answers alongside the traditional, keyword-based matches.

The Editor's Choice feature assures that common -- or easily recognisable -- queries
lead to a mix of mechanically-chosen and editorially-curated links,
with the intention this provides a balance between the large-scale
matching of a search engine with some human insight into content
most users are likely to be looking for -- intelligence that is not
so easily bestowed upon a search engine.

Howver, each and every link promoted within this feature is hand-chosen by editorial
staff and much manual effort is needed to keep promoting new concepts
as they appear and to remove pages as they get superseded.

