\section{Rich Snippets}

The overall information architecture, design and user experience of
the results page are out of scope for this proposal, but it is likely
there could be some improvements in this area, particulary around making
the web page itself responsive\cite{marcotte2010responsive}. A separate
study might consider reviewing the use cases of BBC Search from
the viewpoint of website visitors and consider the relevant
\emph{search modes} and \emph{mode chains}, which could lead to
an information architecture that would drive metadata requirements.

In this proposal, the focus is to propose what \emph{could} be
achieved in an automated way when the principles of linked data
are introduced to the search application. Ultimately, proposed
changes to the search results page should ideally be supported
by valid analysis from a user experience and information
architecture stance.

One improvement may be to introduce the rich snippets as described
in section~\ref{enhancing-results}. The work of Haas et
al.\cite{haas2011enhanced} included adding the snippets in the presentation
layer by triggering one of a list of plugins that will alter the
presentation of the data. The indexing process improvements
described in section~\ref{sec:ld-indexing} should provide the
additional structured data to drive a series of rich snippet templates
for different types of content item.

One drawback with creating a template per ``type'' of item is that a
developer has to create each template individually and also add templates
and redeploy the application to add a new type when it emerges.
The RDF model appears to encourage flexibility through
\emph{schemaless} data structures, which could limit the ability to
trigger fixed templates on some types of objects.
